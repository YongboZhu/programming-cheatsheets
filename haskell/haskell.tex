\documentclass[a4paper, twoside, 8pt]{extarticle}
\usepackage[
  left=1.2cm,
  right=1.2cm,
  top=2.25cm,
  bottom=1.25cm]{geometry}
\usepackage{multicol}

% extensive control of page headers and footers
\usepackage{fancyhdr} 
\makeatletter
\fancypagestyle{mypagestyle}
{\newpage \fancyfoot[C]{} \renewcommand{\footrulewidth}{0pt}}
\makeatother
\pagestyle{mypagestyle}
\headsep 5pt      

% highlighted source code
\usepackage{minted} 
\usemintedstyle{pastie}

% selects alternative section titles
\usepackage[compact]{titlesec} 
\titlespacing{\section}{0pt}{\parskip}{0}
\titlespacing{\subsection}{0pt}{\parskip}{0}
\titlespacing{\subsubsection}{0pt}{\parskip}{0}

% custom font
\usepackage{Alegreya}
% provides an extended symbol collection
\usepackage{amssymb} 
\usepackage{hyperref}
% reference last page
\usepackage{lastpage}
% layout with zero \parindent, non-zero \parskip 
\usepackage{parskip} 
% driver-independent color extensions
\usepackage{xcolor} 
% compact list environment
\usepackage{paralist}
\usepackage{microtype}

\newcommand{\manualbreak}{\vspace*{\fill}\columnbreak}

\usepackage[utf8]{inputenc}
\usepackage[T1]{fontenc}

\begin{document}
\renewcommand{\footrulewidth}{0.4pt}
\fancyhead[LE,LO]{Haskell -- quick reference (page \thepage/\pageref{LastPage})}
\fancyhead[RO,RE]{source: \url{https://git.io/v9d39}} 
\fancyfoot[RF]{author: Remigiusz Suwalski,\\ date: \today}
\fancyfoot[LF]{}

\begin{multicols*}{3}
\section{Prelude}
\subsection{List operations}
\mintinline{haskell}{map}: applies a function to each element of a list.

\mintinline{haskell}{(++)}: appends two lists.

\mintinline{haskell}{filter}: list of elements that satisfy the predicate.

\mintinline{haskell}{head}: first element of a list.

\mintinline{haskell}{last}: last element of finite, non-empty list.

\mintinline{haskell}{tail}: elements after the head of a list.

\mintinline{haskell}{init}: all elements except the last one.

\mintinline{haskell}{null}: tests whether a structure is empty.

\mintinline{haskell}{length}: size of a finite structure.

\mintinline{haskell}{(!!)}: list index (subscript) operator.

\mintinline{haskell}{reverse}: elements of a finite list in reverse order.

\subsubsection{Special folds}
\mintinline{haskell}{and}: the conjunction ($\wedge$) of a container of Bools.

\mintinline{haskell}{or}: the disjunction ($\vee$) of a container of Bools.

\mintinline{haskell}{any}: does any element satisfies the predicate?

\mintinline{haskell}{all}: do all elements satisfy the predicate?

\mintinline{haskell}{concat}: concatenates elements of a container of lists.

\mintinline{haskell}{concatMap}: maps a function over all the elements of a container and concatenate the resulting lists.

\subsubsection{Building lists with scans}
\mintinline{haskell}{scanl}: list of successive reduced values from the left, similar to \mintinline{haskell}{foldl}: $[z, z * x_1, (z * x_1) * x_2, \ldots]$.

\mintinline{haskell}{scanl1}: variant of \mintinline{haskell}{scanl} with no starting value.

\mintinline{haskell}{scanr}: right-to-left dual of \mintinline{haskell}{scanl}

\mintinline{haskell}{scanr1}: variant of \mintinline{haskell}{scanr} with no starting value.

\subsubsection{Building infinite lists}
\mintinline{haskell}{íterate}: $\infty$ list of repeated function applications.

\mintinline{haskell}{repeat}: $\infty$ list from a single element.

\mintinline{haskell}{replicate}: finite list from a single element.

\mintinline{haskell}{cycle}: ties a finite list into a circular one.

\subsubsection{Sublists}
\mintinline{haskell}{take}: prefix of fixed length.

\mintinline{haskell}{drop}: suffix after prefix of fixed length.

\mintinline{haskell}{splitAt}: a tuple equivalent to \mintinline{haskell}{take} and \mintinline{haskell}{drop}.

\mintinline{haskell}{takeWhile}: the longest prefix satisfying a predicate.

\mintinline{haskell}{dropWhile}: the suffix remaining after \mintinline{haskell}{takeWhile}. 

\mintinline{haskell}{span}: a tuple from \mintinline{haskell}{takeWhile} and \mintinline{haskell}{dropWhile}.

\mintinline{haskell}{break}: \mintinline{haskell}{break p == span (not . p)}

\subsubsection{Searching lists}
\mintinline{haskell}{notElem}: negation of \mintinline{haskell}{elem}.

\mintinline{haskell}{lookup}: looks up a key in an association list.

\subsubsection{Zipping and unzipping lists}
\mintinline{haskell}{zip}: list of corresponding pairs from two lists.

\mintinline{haskell}{zip3}: analogous to \mintinline{haskell}{zip}, takes three lists.

\mintinline{haskell}{zipWith}: is a  \mintinline{haskell}{zip} with custom zupping function.

\mintinline{haskell}{zipWith3}: analogous to \mintinline{haskell}{zipWith}, takes three lists.

\mintinline{haskell}{unzip}: transforms a list of pairs into two lists.

\mintinline{haskell}{unzip3}: analogous to \mintinline{haskell}{unzip}, returns three lists.

\subsubsection{Functions on strings}
\mintinline{haskell}{lines}: breaks a string up into a list of lines.

\mintinline{haskell}{words}: breaks a string up into a list of words.

\mintinline{haskell}{unlines}: an operation inverse to \mintinline{haskell}{lines}.
 
\mintinline{haskell}{unwords}: an operation inverse to \mintinline{haskell}{words}.

\end{multicols*}
\end{document}

\mintinline{haskell}{}