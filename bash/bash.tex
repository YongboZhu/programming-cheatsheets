\documentclass[a4paper, twoside, 8pt]{extarticle}
\usepackage[
    left=1.2cm,
    right=1.2cm,
    top=2.25cm,
    bottom=1.25cm]{geometry}
\usepackage{multicol}

\usepackage{fancyhdr} % extensive control of page headers and footers
\makeatletter
\fancypagestyle{mypagestyle}
{\newpage \fancyfoot[C]{} \renewcommand{\footrulewidth}{0pt}}
\makeatother
\pagestyle{mypagestyle}
\headsep 5pt            

\usepackage{minted} % highlighted source code
\usemintedstyle{pastie}

\usepackage[compact]{titlesec} % selects alternative section titles
\usepackage{Alegreya}
\usepackage{amssymb} % provides an extended symbol collection
\usepackage{hyperref}
\usepackage{lastpage} % reference last page
\usepackage{parskip} % layout with zero \parindent, non-zero \parskip
\usepackage{xcolor} % driver-independent color extensions

\newcommand{\manualbreak}{\vspace*{\fill}\columnbreak}
\newcommand{\cmdblack}{$\blacksquare$}
\newcommand{\cmd}{$\bigstar$}
 
\usepackage{enumitem} % control layout of itemize, enumerate, description
\newenvironment{enumx} {
	\begin{enumerate}[leftmargin=*]
	\setlength{\topsep}{0pt}
	\setlength{\itemsep}{0pt}
	\setlength{\parskip}{0pt}
	\setlength{\parsep}{0pt}
	}
{\end{enumerate}}

\newenvironment{itemx}
	{\begin{itemize}[leftmargin=*,noitemsep,topsep=0pt]}{\end{itemize}}

\usepackage[utf8]{inputenc}
\usepackage[T1]{fontenc}

\begin{document}
\renewcommand{\footrulewidth}{0.4pt}
\fancyhead[LE,LO]{Linux/Bash -- cheat sheet (page \thepage/\pageref{LastPage})}
\fancyhead[RO,RE]{source: \url{https://git.io/vMQlp}} 
\fancyfoot[RF]{author: Remigiusz Suwalski, date: \today}
\fancyfoot[LF]{\cmd \, denotes a command, \cmdblack \, a coreutils command, \texttt{f} -- a flag}

\begin{multicols}{3}
\textbf{Bash}, a command line interface for interacting with the operating system, was created in the 1980s.
Other popular shells are zsh and fish.

\section{Programming in Bash}
\subsection{Shebang}
The shebang (\texttt{\#!}) at the head of a script indicates an
interpreter for execution, as in \texttt{\#!/bin/bash}.
Lines starting with a \texttt{\#} (with the exception of shebang)
are comments and thus won't be executed.

\subsection{Quoting and literals}
\textbf{Single quotes} \texttt{''} preserve the literal value of characters enclosed within them.
A single quote may not appear between single quotes, even when escaped, but may appear between \textbf{double quotes} \texttt{""}.

They work similarly, with an exception that the shell expands any variables that appear within them.

\subsection{Variables}
\textbf{Variable} names are case sensitive.
They can contain digits and underscores as well,
but a name starting with a digit is not allowed.
Example:
\begin{minted}{bash}
var="kind"
echo ${var}ness # kindness
\end{minted}

Special variables:
\begin{compactenum}
    \item \texttt{\$0}: name of the script itself.
    \item \texttt{\$1}, \texttt{\$2}, \texttt{\$3}, \ldots: the first, second, etc. argument. \texttt{shift} removes first argument and advances rest of them forward.
    \item \texttt{\$*} and \texttt{\$@} denote all the positional parameters.
    \item \texttt{\$\#}: the number of positional parameters
    \item \texttt{\$?}: exit status of last executed command.
    \item \texttt{\$\$}: the process ID of the shell.
    \item \texttt{\$!}: the process ID of last executed command.
\end{compactenum}

To read a line of input, use \texttt{read} shell built-in.
\begin{compactenum}
\item [\cmdvar] \textbf{read} reads a line from the stdin: % and splits it into fields:
\item [\texttt{n}] returns after reading $n$ characters,
\item [\texttt{n}] displays a prompt.
\end{compactenum}

\subsection{Expansions}
''After the command has been split into tokens, these tokens or words are expanded or resolved. There are eight kinds of expansion performed, which we will discuss in the next sections, in the order that they are expanded.''

\subsubsection{Brace expansion}
Brace expansion is used when we need to generate all possible string combinations.
Both of the commands produce the same output:
\begin{minted}{bash}
echo {I,really,love,dots}.
echo I. really. love. dots.
\end{minted}

\textbf{Warning}: it does not expand the variables (\texttt{\$var}), which is done later, but supports ranges (sequences) of characters:
\begin{minted}{bash}
echo {a..t}
a b c d e f g h i j k l m n o p q r s t
\end{minted}

and (maybe zero paded or with an increment rate) integers, assuming the Bash version is 4 or newer:

\begin{minted}{bash}
echo {01..10..1}.~
01. 02. 03. 04. 05. 06. 07. 08. 09. 10.
\end{minted}

There is a tilde expansion as well.
The expressions \texttt{$\sim$} and \texttt{$\sim$<user>} expand to the home directory of the current (or given) user.

\vfill\null
\columnbreak

\subsubsection{Parameter expansion}
\begin{enumerate}
\item \mintinline{bash}{${var^}}, \mintinline{bash}{${var,}} convert first character to upper and lowercase. \mintinline{bash}{${var^^}}, \mintinline{bash}{${var,,}} do the same to all characters.

\mintinline{bash}{${var~}}, \mintinline{bash}{${var~~}} are undocumented now: they reverse the case.

In case of the array expansion, every expanded element changes case, no matter what.

\item \mintinline{bash}{${var#pattern}} removes the \texttt{pattern} from the beginning of the string, if possible.

It's greedy variant is \mintinline{bash}{${var##pattern}}.

\mintinline{bash}{${var%pattern}} and \mintinline{bash}{${var%%pattern}} do the same, but  from the end of the string.

Application: extracting parts of a filename.

\item \mintinline{bash}{${var/pattern/string}} performs a single search and replace operation.

\mintinline{bash}{${var//pattern/string}} searches for all occurrences of the pattern and replaces them.

\item \mintinline{bash}{${#var}} returns length of the string.

\item \mintinline{bash}{${var:offset:length}} skips first \texttt{offset} characters from \texttt{var} and truncates the output to given length.
\texttt{:length} may be skipped.

Negative values separated with extra space are accepted.

\item \mintinline{bash}{${var:-value}} uses a default value, if \texttt{var} is empty or unset.

\mintinline{bash}{${var:=value}} does the same, but performs an assignment as well.

\mintinline{bash}{${var:+value}} uses an alternative value if \texttt{var} isn't empty or unset!
\end{enumerate}

\subsubsection{Command substitution}
To execute commands in a subshell and then pass their standard output, use \mintinline{bash}{$( commands )}.

\subsubsection{Arithmetic expansion}
The arithmetic expression \mintinline{bash}{$(( ... ))} is evaluated and expands to the result.
Bash guarantees that the output will be a one-word integer.

\subsubsection{Process substitution}
This kind of substitution: \mintinline{bash}{<( ... )} and \mintinline{bash}{>( ... )} (not specified by POSIX!), where input or output of a command appears as a temporary file, is performed simultaneously with the following: arithmetic and parameter expansions, command substitution.


\subsection{Streams}
There are always three default files open:
\begin{compactenum}
\item \emph{stdin} (the keyboard, file descriptor 0),
\item \emph{stdout} (the screen, file descriptor 1) and
\item \emph{stderr} (error messages output, file descriptor 2).
\end{compactenum}

These \textbf{streams} can be \textbf{redirected}:
\begin{compactenum}
\item \texttt{cmd > file} redirects to a file (overwrites),
\item \texttt{cmd >{}> file} appends instead,
\item \texttt{m>n} (or \texttt{m>\&n}) redirects a file descriptor to a file
(or another file descriptor),
\item \texttt{\&>file} redirects stdout and stderr to a file,
\item \texttt{:> file} truncates file to zero length,
\item \texttt{|} (pipe) serves as a command chaining tool.
\end{compactenum}

Here document is a section of a source code file that is treated as if it were a separate file:

\begin{minted}{bash}
cat <<EOF > /path/to/your/file
   This line will write to the file.
EOF
\end{minted}

%\vfill\null
%\columnbreak

\subsection{Control flow statements}
The one-line constructs \texttt{\&\&} and \texttt{||} work not like and, or ($\wedge$, $\vee$), but the \emph{if -- then -- else} statement.

\subsubsection{Conditionals}
Here at least one statement must be specified inside every block,
but one can use a single colon (:) as a null statement to avoid
rewriting the code.

\begin{minted}{bash}
if condition; then
  commands
elif second_condition; then
  some_commands
else
  other_commands	
fi

select word in "Bash" "Haskell" "Python"
do
  echo "Your language is $word".
done
\end{minted}

There is also a case instruction:
\begin{minted}{bash}
case $language in
  bash)
    echo "Bourne Again Shell!"
  ;;
  python|haskell)
    echo "Python or Haskell!"
    exit 1
  ;;
  *)
    echo "Unknown language!"
  ;; # optional
esac
\end{minted}

\subsubsection{Testing conditions}
Remember that \texttt{test} command follows symbolic links (except for the \texttt{-h} test).
\begin{compactenum}
\item \textbf{File tests}:
\begin{compactenum}
    \item \texttt{-e} file exists,
    \texttt{-s} file is nonempty,
    \item \texttt{-d} directory,
    \texttt{-f} regular file,
    \texttt{-h} symlink,
    \item \texttt{-b} block device,
    \texttt{-c} character device,
    \item \texttt{-p} named pipe,
    \texttt{-S} socket.
\end{compactenum}
\item \textbf{File permissions}:
\begin{compactenum}
    \item \texttt{-r} readable,
    \texttt{-w} writable,
    \texttt{-x} executable,
    \item \texttt{-u} setuid,
    \texttt{-g} setgid,
    \texttt{-k} sticky bit.
\end{compactenum}
\item \textbf{String tests}: \texttt{-z} empty, \texttt{-n} nonempty.
\item \textbf{Arithmetic tests}:
\begin{compactenum}
  \item \texttt{-eq} $=$,
  \texttt{-ne} $\neq$,
  \item \texttt{-lt} $<$,
  \texttt{-gt} $>$,
  \item \texttt{-le} $\le$,
  \texttt{-ge} $\ge$.
\end{compactenum}
\end{compactenum}

\subsubsection{Loops}
\begin{minted}{bash}
for var in "the first" "the second"; do
  echo "${var}"
done

for (( i = 1; i <= 10; i++ )); do
  echo "i = ${i}."
done # C-style

while read myline; do
  echo "It says ${myline}"
done < some_file
\end{minted}

As Bash Guide for Beginners by M. Garrels says:
\begin{compactenum}
\item the \texttt{break} statement is used to
exit the current loop before its normal ending.
\item the \texttt{continue} statement resumes iteration
of an enclosing while, until, select or for loop.
\end{compactenum}

\vfill\null
\columnbreak


% special files: /dev/null, /dev/zero, /proc/...

% special meaning of < > ; | * ? - one has to escape them
% ' '
% " "


\subsection{Regular expressions}
\begin{itemx}
\item POSIX character classes:
\begin{itemx}
\item \texttt{[:alnum:]} $=$ \texttt{[a-zA-Z0-9]}
\item \texttt{[:alpha:]} $=$ \texttt{[a-zA-Z]}
\item \texttt{[:ascii:]} $=$ \texttt{[\textbackslash{}x00-\textbackslash{}x7F]}
\item \texttt{[:blank:]} $=$ \texttt{[ \textbackslash{}t]}
\item \texttt{[:cntrl:]} $=$ \texttt{[\textbackslash{}x00-\textbackslash{}x1F\textbackslash{}x7F]}
\item \texttt{[:digit:]} $=$ \texttt{[0-9]}
\item \texttt{[:graph:]} $=$ \texttt{[\textbackslash{}x21-\textbackslash{}x7E]}
\item \texttt{[:lower:]} $=$ \texttt{[a-z]}
\item \texttt{[:print:]} $=$ \texttt{[\textbackslash{}x20-\textbackslash{}x7E]}
%\item \texttt{[:punct:]} $=$ \texttt{	Punctuation and symbols.	[!"\#$%&'()*+,\-./:;<=>?@\[\\\]^_`{|}~]	\p{P}		\p{Punct}
\item \texttt{[:space:]} $=$ \texttt{[ \textbackslash{}t\textbackslash{}r\textbackslash{}n\textbackslash{}v\textbackslash{}f]}
\item \texttt{[:word:]} $=$ \texttt{[A-Za-z0-9\_]}
\item \texttt{[:xdigit:]} $=$ \texttt{[A-Fa-f0-9]}
\end{itemx}
\item Repetitions:
\begin{itemx}
\item \texttt{*}: $0$ or more, \texttt{+}: $1$ or more, \texttt{?}: $0$ or $1$,
\item \texttt{\{a, b\}}: at least $a$, at most $b$.
\end{itemx}
\item Anchors:
\begin{itemx}
\item \texttt{\textasciicircum}: start of line,
\item \texttt{\$}: end of line, 
\item \texttt{\textbackslash{}<}: start of word, 
\item \texttt{\textbackslash{}>}: end of word.
\end{itemx}
\item Other:
\begin{itemx}
\item \texttt{one|two}: one or two,
\item \texttt{(one)}: a group,
\item \texttt{\$n}: $n$th group,
\item \texttt{[abcd]}, \texttt{[a-d]}: ranges,
\item \texttt{[\textasciicircum{}abcd]}: negation (not \texttt{[abcd]}).
\end{itemx}
\end{itemx}


\section{Text processing: grep, sed, awk}
This is an expanded description of three powerful text processing tools:
\texttt{grep}, \texttt{sed} and \texttt{awk}.

See also: \url{https://beyondgrep.com/}, \texttt{ack}.

\subsection{grep -- pattern search enginge}
The ed command \texttt{g/re/p} was used to \textbf{g}lobally search a \textbf{r}egular \textbf{e}xpression and \textbf{p}rint.
\begin{compactenum}
\item [\cmdvar] \textbf{grep} prints lines matching a pattern:
\item [\texttt{c}] prints a count of matching lines instead,
\item [\texttt{e}] uses a ``regexp'' pattern,
\item [\texttt{f}] obtains patterns from a file,
\item [\texttt{i}] ignores case disctinctions,
\item [\texttt{v}] inverts the sense of matching,
\item [\texttt{w}] selects only lines with whole words matches,
\item [\texttt{n}] prints line numbers as well,
\item [\texttt{A}] prints ``num'' lines of trailing content,
\item [\texttt{B}] prints ``num'' lines of leading content,
\item [\texttt{C}] prints ``num'' lines of both contents,
\item [\texttt{E}] interprets pattern as an extended regexp,
\item [\texttt{P}] interprets pattern as a Perl regexp,
\item [\texttt{R}] reads all files under each directory.
\end{compactenum}

\subsection{sed -- stream editor}
\begin{compactenum}
	\item [\cmdcore] \textbf{sed} filters and transforms text:
	\item [\texttt{-e}] adds a script to the commands to be executed,
	\item [\texttt{-i}] edits files in place,
	\item [\texttt{-n}] suppresses auto- printing of pattern space,
  \item [\texttt{-r}] accepts extended regular expressions.
\end{compactenum}

The simplest usage is \mintinline{bash}{sed 's/foo/bar/g'}
which substitutes (\texttt{s}) strings globally (\texttt{g}).
There are other options, including:
\begin{compactenum}
\item [\texttt{a}] appends line before,
\item [\texttt{d}] deletes line,
\item [\texttt{i}] inserts line before,
\item [\texttt{p}] prints line,
\item [\texttt{w}] writes pattern space to a file.
\end{compactenum}

Default delimiter \texttt{/} can be replaced by any other.
This is useful when regular expression already contains \texttt{/}.
Addresses allow limiting to given line numbers:
\begin{compactenum}
\item \texttt{1-10} first ten lines
\item \texttt{\$} the last line
\item \texttt{10\textasciitilde 2} even lines starting from the 10th.
\end{compactenum}

One can also use regular expressions:
\begin{minted}{bash}
sed -e '/:/s/ /_/g'
\end{minted}
replaces spaces with underscores in lines containing a colon.
Negation may be obtained with \texttt{!s}.

% # d 1d 1,10d

\subsection{awk -- Aho, Weinberger, Kernighan}
\begin{compactenum}
\item [\cmdvar] \textbf{awk} is a language used as a data extraction and reporting tool.
\end{compactenum}

General form of its code:
\begin{minted}{bash}
#! /bin/awk
BEGIN {initialization}
search pattern {actions} # for example:
/word[0-9]/ {gold += $2} # regex
!/word[0-9]/ {counter++} # negation
END {final actions}
\end{minted}

Awk is weakly typed: variables can be treated either as numeric values or strings, which are not represented as one-dimensional arrays of characters!
Important variables include:
\begin{compactenum}
\item \textbf{FS}: field separator (tab and space by default),
\item \textbf{OFS}: output field separator,
\item \textbf{RS}: record separator (new line),
\item \textbf{NR}: number of the current record,
\item \textbf{NF}: number of fields in the current record. 
\end{compactenum}

Numerical functions: \texttt{int}, \texttt{sqrt}, \texttt{exp}, \texttt{log}, \texttt{sin}, \texttt{cos}, \texttt{atan2}, \texttt{rand} (pseudo random from $[0, 1)$), \texttt{srand} (without parameters, uses time of day as a seed).

String/text functions: \texttt{length}, \texttt{split}, \texttt{sprintf}, \texttt{gsub}, \texttt{sub}, \texttt{index}, \texttt{match}, \texttt{tolower}, \texttt{toupper}.

\section{Emacs shortcuts in Bash}
{\small(See \url{http://readline.kablamo.org/emacs.html})}
\begin{compactenum}
\item \texttt{Ctrl A} moves to the start of the line,
\item \texttt{Ctrl E} moves to the end of the line,
\item \texttt{Ctrl U} deletes to the beginning of the line.
\item \texttt{Ctrl K} deletes to the end of the line.
\item \texttt{Ctrl W} deletes to the start of the word.
\item \texttt{Ctrl Y} pastes text from the clipboard.
\item \texttt{Ctrl L} clears the screen.
\item \texttt{ Alt R} undoes all changes to the line.
\item \texttt{Ctrl R} searches incrementally up the history.
\item \texttt{Ctrl XE} invokes an editor to write complex command.
\end{compactenum}


\end{multicols}

\newpage

\begin{multicols}{3}
\section{Unix utilities and shell builtins}
% Based on https://en.wikipedia.org/wiki/Template:Unix_commands

\subsection{File system}
\begin{enumx}
	\item [\cmdblack] \textbf{cat} concatenates and prints files:
	\item [\texttt{A}] shows all nonprinting characters,
	\item [\texttt{b}] numbers nonempty output lines,
	\item [\texttt{n}] numbers all output lines,
	\item [\texttt{s}] suppresses repeated empty output lines.
    % tac -r -s 'x|[^x]'
	\item [\cmdblack] \textbf{tac} does the same in reverse.
	\item [\cmd] \textbf{rev} reverses lines characterwise.
	\item [\cmdblack] \textbf{nl} numbers lines of files:
	\item [\texttt{s}] adds ``string'' after line number,
	\item [\texttt{w}] uses ``number'' columns for line numbers.
\end{enumx}

\begin{enumx}
	\item [\cmdblack] \textbf{chgrp} changes group ownership.
	
	\item [\cmdblack] \textbf{chmod} changes permissions of a file:
	\item [\texttt{ugoa}] permissions of the owner, group, other/all users,
	\item [\texttt{+-=}] adds, removes or sets selected file mode bits,
	\item [\texttt{rwx}] selects file mode bits: read/write/execute (4/2/1).
	
	\item [\cmdblack] \textbf{chown} changes owner of a file.
	
	\item [\cmd] \textbf{umask} sets file mode creation mask.

	\item [\cmdblack] \textbf{touch} changes file timestamps:
	\item [\texttt{a}] only the access time,
	\item [\texttt{m}] only the modification time,
	\item [\texttt{t}] uses custom stamp instead of current time,
	\item [\texttt{c}] does not create files.
\end{enumx}

\begin{enumx}
	\item [\cmdblack] \textbf{shasum} prints or checks SHA message digests:
	\item [\texttt{a}] algorithm: 1, 224, 256, 384, 512, 512224 or 512256,
	\item [\texttt{b}] reads in binary mode,
	\item [\texttt{c}] checks SHA sums read from the ``files''.

	\item [\cmdblack] See also \textbf{cksum} (CRC checksums) and \textbf{md5sum}.
	\item [\cmdblack] \textbf{wc} prints newline, word and byte counts (\texttt{lwc}):
	%\item [\texttt{c}] prints the byte counts,
	%\item [\texttt{l}] prints the newline counts,
	\item [\texttt{m}] prints the character counts,
	%\item [\texttt{w}] prints the word counts.
	\item [\texttt{L}] prints the maximum display width.
\end{enumx}

\begin{enumx}
	\item [\cmdblack] \textbf{dd} converts and copies a file:
	\item [\texttt{if=}] reads from a file instead of standard input,
	\item [\texttt{of=}] writes to a file insteaed of standard output,
	\item [\texttt{bs=}] up to ``bytes'' bytes at a time,
	\item [\texttt{count=}] copies only ``n'' input blocks.
\end{enumx}

\begin{enumx}
	\item [\cmdblack] \textbf{cp} copies files and directories:
%	\item [\texttt{a}] never follows symlinks, preserves all attributes,
	\item [\texttt{b}] makes a backup of each existing destination file,
	\item [\texttt{f}] removes an existing destination file if needed,
	\item [\texttt{i}] prompts before overwrite,
	\item [\texttt{n}] does not overwrite existing files,
	\item [\texttt{L}] always follows symlinks in ``source'',
	\item [\texttt{P}] never follows symlinks in ``source'',
	\item [\texttt{p}] preserves timestamps, mode, ownership,
	\item [\texttt{r}] copies directories recursively,
	\item [\texttt{s}] makes symbolic links instead,
	\item [\texttt{l}] hard links files instead,
	\item [\texttt{t}] copies all ``source'' arguments into ``directory'',
	\item [\texttt{T}] treats ``destination'' as a normal file,
	\item [\texttt{u}] copies only newer source files,
	\item [\texttt{v}] explains what is being done.
	
	\item [\cmdblack] \textbf{mv} moves (renames) files:
	\item [\texttt{b}] makes a backup of each existing destination file,
	\item [\texttt{i}] prompts before overwriting,
	\item [\texttt{f}] does not prompt before overwriting,
	\item [\texttt{n}] does not overwrite existing destination files.
	\item [\texttt{t}] moves all ``source'' arguments into ``directory'',
	\item [\texttt{T}] treats ``destination'' as a normal file,
	\item [\texttt{u}] moves only newer source files,
	\item [\texttt{v}] explains what is being done.
	
	\item [\cmdblack] \textbf{rm} removes files or directories:
	\item [\texttt{f}] never prompts,
	\item [\texttt{i}] always prompts,
	\item [\texttt{r}] removes directories and their contents.

	\item [\cmdblack] See also \textbf{rmdir} (directories removal) and \textbf{shred}.

	\item [\cmdblack] \textbf{mkdir} makes directories 
	(\texttt{mkdir p}: with parents as needed, no error if existing).
\end{enumx}

\begin{enumx}
	\item [\cmdblack] \textbf{df} reports file system disk space usage:
	\item [\texttt{h}] prints size in powers of 1024,
	\item [\texttt{i}] list inode information instead of block usage,
	\item [\texttt{t}] limits listing to file systems of given type,
	\item [\texttt{x}] limits listing to file systems not of given type,
	\item [\texttt{T}] prints file systems types.
	
	\item [\cmdblack] \textbf{du} estimates file space usage:
	\item [\texttt{a}] writes counts for all files, not just directories,
	\item [\texttt{c}] produces a grand total,
	\item [\texttt{d}] the depth at which summing should occur,
	\item [\texttt{h}] prints sizes in human readable format,
	\item [\texttt{s}] diplays only a total,
	\item [\texttt{X}] excludes files that match pattern.
\end{enumx}

\begin{enumx}
	\item [\cmd] \textbf{file} determines file type.
\end{enumx}

\begin{enumx}
	\item [\cmd] \textbf{find} searches for files in a directory hierarchy.
	\item Tests:
	\begin{itemx}
	\item \texttt{-name} \hfill base of file name,
	\item \texttt{-iname} \hfill case insensitive name,
	\item \texttt{-group}, \texttt{-user} \hfill ownership
	\item \texttt{-perm 755}, \texttt{-perm /u=x} \hfill permissions
	\item \texttt{-size +5M -1G} \hfill size between 5MB and 1GB
	\item \texttt{-amin -60} \hfill accessed in last hour
	\item \texttt{-cmin}, \texttt{-mmin}: \hfill created, modified,
	\item \texttt{-mtime +7} \hfill modified over a week ago
	\item \texttt{-type d} \hfill directories only,
	\item \texttt{-type f} \hfill files only,
	\item \texttt{-empty} \hfill empty files or directories only,
	\end{itemx}
	\item Example (deletes files larger than 5 megabytes): 
	\begin{itemx}
	\item \texttt{find . -size +5M -exec rm -f {} \textbackslash{};}
	\end{itemx}
\end{enumx}

\begin{enumx}
	\item [\cmd] \textbf{fsck} checks and repairs a Linux filesystem:
	\item [\texttt{a}] automatically repairs (without any question!),
	\item [\texttt{t}] specifies the type(s) of filesystem to be checked,
	\item [\texttt{A}] tries to check all filesystems in one run,
	\item [\texttt{M}] skips mounted filesystems,
	\item [\texttt{R}] skips the root filesystem.
\end{enumx}

\begin{enumx}
	\item [\cmdblack] \textbf{ln} makes hard links between files
	(not directories; only in the same file system):
	\item [\texttt{s}]  makes symbolic links instead.
\end{enumx}

\begin{enumx}
	\item [\cmdblack] \textbf{ls} lists directory contents:
	\item [\texttt{a}] does not ignore entries starting with dot, 
	\item [\texttt{F}] appends indicator to entries, 
	\item [\texttt{h}] prints human readable sizes, 
	\item [\texttt{i}] prints the index number of each file, 
	\item [\texttt{l}] prints permissions, number of hard links, owner, group, size, last-modified date as well, 
	\item [\texttt{r}] reverses order while sorting,
	\item [\texttt{R}] lists subdirectories recursively, 
	\item [\texttt{S}] sorts by file size (largest first), 
	\item [\texttt{t}] sorts by modification time (newest first), 
	\item [\cmd] \textbf{tree} lists tree-like contents of directories.
\end{enumx}

\begin{enumx}
	\item [\cmd] \textbf{mount} mounts a filesystem.
\end{enumx}

\begin{enumx}
	\item [\cmdblack] \textbf{pwd} prints name of current directory.
\end{enumx}

\begin{enumx}
	\item [\cmd] \textbf{tar} stores and extracts files from a disk archive:
	\item [\texttt{c}] creates a new archive,
	\item [\texttt{x}] extracts files,
	\item [\texttt{t}] lists the contents of an archive,
	\item [\texttt{v}] verbosely lists files processed,
	\item [\texttt{j}] bzip2 compression,
	\item [\texttt{z}] uses zip/gzip (gz compression),
	\item [\texttt{f}] uses archive file or device (???),
	\item [\texttt{k}] does not replace existing files when extracting.
\end{enumx}


\begin{enumx}
	\item [\cmdblack] \textbf{pv} monitors the progress of data through a pipe.
\end{enumx}

\begin{enumx}
	\item [\cmdblack] \textbf{tee} duplicates pipe content: % (named after the T-splitter used in plumbing) 
	\item [\texttt{a}] appends to the given files, does not overwrite,
	\item [\texttt{i}] ignores interrupts.
\end{enumx}

\begin{enumx}
	\item [\cmd] Missing: \textbf{cmp}, \textbf{fuser}, \textbf{pax}, \textbf{type}.
\end{enumx}


\subsection{Processes}
\begin{enumx}
	\item [\cmdblack] \textbf{chroot} changes the root directory 
	for the current running process and their children.
\end{enumx}

\begin{enumx}
	\item [\cmd] \textbf{at} schedules commands to be executed once, 
	at a particular time in the future: it accepts times of the form 
	\texttt{HH:MM}, \texttt{midnight}, \texttt{noon} or \texttt{teatime}; 
	\texttt{MMDD[CC]YY}, \texttt{MM/DD/[CC]YY}, \texttt{DD.MM.[CC]YY} or 
	\texttt{[CC]YY-MM-DD} (the specification of a date 
	must follow the specification of the time of day).
	You can also give times like \texttt{now + 3 hours}.
\end{enumx}

\begin{enumx}
	\item [\cmd] \textbf{bg} resumes suspended jobs in the background.
	\item [\cmd] \textbf{fg} resumes suspended jobs in the foreground.
	\item [\cmd] \textbf{jobs} lists the active jobs.
	\item [\cmd] \textbf{command \&} runs command in the background.
	% || versus && (error versus success)
\end{enumx}

\begin{enumx}
	\item [\cmd] \textbf{cron}: a daemon executing scheduled commands.
	\item [\cmd] \textbf{crontab} maintain individual users' crontab files.
\end{enumx}

\begin{enumx}
	\item [\cmd] \textbf{kill} sends a \texttt{TERM} signal to a process.
	\item [\cmd] \textbf{killall} kills processes by name.
\end{enumx}

\begin{enumx}
	\item [\cmd] \textbf{ps} reports a snapshot of the current processes:
	\item [\texttt{e}] selects all processes,
	\item [\texttt{f}] does full-format listing,
	\item [\texttt{C}] selects processes by command name,
	\item [\texttt{p}] selects processes by PID,
	\item [\texttt{u}] selects processes by EUID or name. 
	\item [\cmd] \textbf{pstree} displays a tree of processes.
\end{enumx}

\begin{enumx}
	\item [\cmd] \textbf{nice} changes process priority.
\end{enumx}

\begin{enumx}
	\item [\cmd] \textbf{pgrep}, \textbf{pkill} looks up or signals 
processes based on name and other attributes.
\end{enumx}

\begin{enumx}
	\item [\cmd] \textbf{time} runs programs and summarizes system resource usage. 
\end{enumx}

\begin{enumx}
	\item [\cmd] \textbf{top} displays linux processes.
	\item [\cmd] See also: \textbf{htop} (Hisham top).
\end{enumx}


\subsection{User environment}
\begin{enumx}
	\item [\cmd] \textbf{clear} clears the terminal screen.
	\item [\cmd] \textbf{env} runs a program in a modified environment.
	\item [\cmd] \textbf{exit} terminates the calling process.
	\item [\cmd] \textbf{finger} looks up user information.
	\item [\cmd] \textbf{history}  displays the history list. %with line numbers.
%	\item [\cmd] \textbf{logname} prints user's login name.
	\item [\cmd] \textbf{mesg} displays messages from other users.
\end{enumx}

\begin{enumx}
	\item [\cmd] \textbf{passwd} changes user password:
	\item [\texttt{d}] deletes an account's password (makes it empty),
	\item [\texttt{e}] expires an account's password,
	\item [\texttt{n}] minimum days to change password,
	\item [\texttt{w}] warning days before password expire,
	\item [\texttt{x}] maximum days a password remains valid.
	\item [\cmd] \textbf{pwgen} generate pronounceable passwords:
	\item [\texttt{s}] generates hard to memorize passwords,
	\item [\texttt{y}] includes special characters,
	\item [\texttt{n}] includes numbers,
	\item [\texttt{N}] generates ``num'' passwords
\end{enumx}

\begin{enumx}
        \item [\cmd] \textbf{su} changes user ID or becomes superuser.
        \item [\cmd] \textbf{sudo} executes a command as superuser:
        \item [\texttt{u}] as a different user.
\end{enumx}


\begin{enumx}
	\item [\cmd] \textbf{hostname} shows/sets the host name:
	\item [\texttt{i}] displays the network address.
	\item [\cmd] \textbf{uname} prints system information:
	\item [\texttt{a}] all information, in the following order:
	\item [\texttt{s}] the kernel name,
	\item [\texttt{n}] the network node hostname,
	\item [\texttt{r}] the kernel release,
	\item [\texttt{v}] the kernel version,
	\item [\texttt{m}] the machine hardware name,
	\item [\texttt{p}] the processor type,
	\item [\texttt{i}] the hardware platform,
	\item [\texttt{o}] the operating system.
\end{enumx}

\begin{enumx}
	\item [\cmd] \textbf{uptime}: how long has the system been running?
\end{enumx}

\begin{enumx}
	\item [\cmd] \textbf{wall} writes a message to all users,
	\item [\cmd] \textbf{write} sends a message to another user. 
\end{enumx}

\begin{enumx}
	\item [\cmd] \textbf{who} shows who is logged on,
	\item [\cmd] \textbf{w} does the same and shows what they are doing,
	\item [\cmd] \textbf{whoami} prints effective userid.
\end{enumx}



\subsection{Text processing}
\renewcommand\theFancyVerbLine{\normalsize\arabic{FancyVerbLine}}

\begin{enumx}
	\item [\cmd] \textbf{awk} is a pattern scanning / processing language,
	a pseudo-C interpretor.
	Sample code:
\begin{minted}[linenos, numbersep=3pt, frame=lines, framesep=1mm]{bash}
BEGIN {print "- Start -"}
/word/ {print NR ")" $1, $2}
END {print "- End -"}
\end{minted}

\item [] Examples of conditions:
\begin{enumx}
	\item \texttt{/word[0+9]+/}: regular expressions
	\item \texttt{!/word[0+9]+/}: regexes inverted
	\item \texttt{$\sim$} and \texttt{!$\sim$}: matches / does not match.
	\item \texttt{length(\$0) > 18}.
\end{enumx} 

\item [] Important variables:
\begin{enumx}
	\item FS: field separator (tab and space by default),
	\item OFS: output field separator,
	\item RS: record separator (new line),
	\item NR: number of the current record,
	\item NF: number of fields in the current record.
\end{enumx} 

\item [\cmd] \textbf{grep} prints lines matching a pattern:
\item [\texttt{c}] prints a count of matching lines instead,
\item [\texttt{e}] uses a ``regexp'' pattern,
\item [\texttt{f}] obtains patterns from a file,
\item [\texttt{i}] ignores case disctinctions,
\item [\texttt{v}] inverts the sense of matching,
\item [\texttt{w}] selects only lines containing matches that form whole words,
\item [\texttt{n}] prints line numbers as well,
\item [\texttt{A}] prints ``num'' lines of trailing content,
\item [\texttt{B}] prints ``num'' lines of leading content,
\item [\texttt{C}] prints ``num'' lines of both contents,
\item [\texttt{R}] reads all files under each directory.
\item [\cmd] \textbf{sed}: a stream editor filtering/transforming text.
\end{enumx}

\begin{enumx}
	\item [\cmdblack] \textbf{comm} compares two sorted files line by line.
	\item [\cmdblack] \textbf{shuf} generates random permutations:
	\item [\texttt{e}] treats each ``arg'' as an input line,
	\item [\texttt{i}] treats each number .. through .. as an input line, 
	\item [\texttt{n}] outputs at most ``count'' lines,
	\item [\texttt{r}] output lines can be repeated (with \texttt{-n}).
	\item [\cmdblack] \textbf{sort} sorts lines of text files:
	\item [\texttt{c}] checks for sorted input,
	\item [\texttt{f}] folds lower case to upper case characters,
	\item [\texttt{g}] compares general numerical values,
	\item [\texttt{h}] compares human readable numbers,
	\item [\texttt{k}] sorts via a key,
	\item [\texttt{n}] compares string numerical values,
	\item [\texttt{r}] reverses the results,
	\item [\texttt{s}] stabilizes the sort.
	\item [\cmdblack] \textbf{tsort} performs topological sort.
	\item [\cmdblack] \textbf{uniq} omits repeated lines:
	\item [\texttt{c}] prefixes lines by the number of occurences,
	\item [\texttt{d}] only prints duplicate lines, one for each group,
	\item [\texttt{f}] avoids comparing first fields,
	\item [\texttt{i}] ignores differences in case,
	\item [\texttt{s}] avoids comparing first characters,
	\item [\texttt{w}] compares no more than $n$ characters.
\end{enumx}

\begin{enumx}
	\item [\cmdblack] \textbf{cut} prints selected parts of lines:
	\item [] \texttt{-}\texttt{-}\texttt{complement} complements the selection,
	\item [\texttt{c}] selects only these characters,
	\item [\texttt{d}] uses ``delim'' instead of Tab for field delimeter,
	\item [\texttt{f}] selects only these fields,
	\item [\texttt{s}] does not print lines not containing delimeters.
	\item [\cmdblack] \textbf{join} joins lines of two files on a common field.
	\item [\cmdblack] \textbf{paste} merges lines of files.
	\item [\texttt{d}] reuses characters from ``list'' instead of tabs,
	\item [\texttt{s}] pastes one file at a time, not in parallel.
	\item [\cmdblack] \textbf{tr} translates or deletes characters:
	% \item \texttt{tr abc xyz} changes \texttt{a} to \texttt{x}, $\ldots$,
	\item [c] uses the complement of ``set1'',
	\item [d] deletes characters, does not translate,
	\item [s] replaces each sequence of a repeated character that is listed 
	in the last specified ``set'' with a single occurrence of that character.
\end{enumx}

\begin{enumx}
	\item [\cmd] \textbf{diff} compares files line by line:
	\item [\texttt{y}] outputs in two columns,
	\item [\texttt{i}] ignores case differences,
	\item [\texttt{w}] ignores all white space.
	% E Z b B
\end{enumx}

\begin{enumx}
	\item [\cmd] \textbf{fmt} is a simple optimal text formatter, 
	\item [\cmd] \textbf{fold} wraps each line to fit in specified width.
\end{enumx}

\begin{enumx}
	\item [\cmdblack] \textbf{head} outputs the first (last) part of files:
	\item [\texttt{c}] the first ``num'' bytes,
	\item [\texttt{n}] the first ``num'' lines,
	\item [\cmdblack] \textbf{tail} the last ``num'' bytes:
	\item [\texttt{c}] the last ``num'' bytes,
	\item [\texttt{n}] the last ``num'' lines,
	\item [\texttt{f}] outputs appended data as the file grows,
	\item [\texttt{s}] sleeps for ``n'' seconds between iterations. 
	\item [\cmdblack] \textbf{split} splits a file into pieces:
	\item [\texttt{a}] generates suffixes of length ``n'' (default 2),
	\item [\texttt{b}] puts ``size'' bytes per output file,
	\item [\texttt{d}] uses numeric (not alphabetic) suffixes,
	\item [\texttt{l}] puts ``number'' lines/records per output file,
	\item [\texttt{n}] generates ``chunks'' output files.
	\item [\cmdblack] See also: \textbf{csplit}.
\end{enumx}

\begin{enumx}
	\item [\cmd] \textbf{more} pages text too large to fit on one screen and 
	allows scrolling down, but not up and therefore is deprecated.
	\item [\cmd] \textbf{less} is an enhanced version of more:
	\item [\texttt{+F}] monitors the tail of a file which is growing.
\end{enumx}

\begin{enumx}
	\item [\cmd] \textbf{vim} is an advanced text editor, 
	too complex to be explained here.
	See also \textbf{emacs}.
\end{enumx}

\begin{enumx}
	\item [\cmd] \textbf{xargs} builds and executes command lines:
	\item [\texttt{0}] takes care of filenames with spaces, backslashes.
	\item [\texttt{I}] replaces occurrences of ``string'' with names read from standard input.
\end{enumx}

\begin{enumx}
	\item [\cmd] \textbf{yes} outputs a string repeatedly until killed.
\end{enumx}


\subsection{Shell builtins}
\begin{compactenum}
	\item [\cmdvar] \textbf{alias} allows a string to be substituted for a word.
	\item [\cmdvar] \textbf{cd} changes the shell working directory:
	\item [\texttt{-}] to the previous directory.
	\item [\cmdvar] \textbf{echo}* displays a line of text:
	\item [\texttt{e}] enables interpretation of backslash escapes,
	\item [\texttt{n}] does not output the trailing newline.
	\item [\cmdvar] \textbf{test} checks file types and compares values.
	\item [\cmdvar] \textbf{unset} unsets a shell variable, removing it from memory and the shell's exported environment.
	\item [\cmdvar] \textbf{wait} waits for process to change state.
\end{compactenum}


\subsection{Networking}
\begin{compactenum}
\item [\cmdvar] \textbf{curl} transfers a URL.
\item [\cmdvar] \textbf{ftp} is a File Transfer Protocol client.
\item [\cmdvar] \textbf{wget} is a non-interactive network downloader.
\item [\texttt{A}, \texttt{R}] specifies lists 	of file suffixes or 
	patterns (when wildcard characters appear) to accept or reject,
\item [\texttt{b}] goes to background immediately after startup,
\item [\texttt{c}] continues getting a partially-downloaded file,
\item [\texttt{m}] turns on options suitable for mirroring: 
	infinite recursion and time-stamping,
\item [\texttt{np}] does not ever ascend to the
	parent directory when retrieving recursively,
\item [\texttt{U}] identifies as ``agent-string'' to the HTTP server.
\item [\texttt{w}] waits the specified number of seconds 
	between the retrievals (see also \texttt{--random-wait}).
\end{compactenum}

\begin{compactenum}
\item [\cmdvar] \textbf{rlogin} starts a terminal session (which is not encrypted!) on a remote host.
\item [\cmdvar] \textbf{ssh} connects \& logs into a (remote) hostname:
\item [\texttt{C}] requests compression of all data,
%\item [\texttt{D}] specifies a local ''dynamic'' application-level port forwarding,
\item [\texttt{i}] uses private key for authentication,
\item [\texttt{p}] selects a port to connect to on the remote host,
\item [\texttt{X}] enables X11 forwarding.
\end{compactenum}

\begin{compactenum}
\item [\cmdvar] \textbf{dig} interrogates DNS name servers.                        
\item [\texttt{x}] performs a simplified reverse lookup. 
\item [\cmdvar] \textbf{host} is a DNS lookup utility.  
\item [\cmdvar] \textbf{nslookup} is (probably) deprecated! Use \textbf{dig} and \textbf{host}.
\end{compactenum}

\begin{compactenum}
\item [\cmdvar] \textbf{arp} manipulates the system ARP cache.
\item [\cmdvar] \textbf{ifconfig} configures a network interface.   
\item [\cmdvar] \textbf{netstat} prints networking subsystem related info: network connections, interface statistics, routing tables, and so on.
\item [\texttt{a}] whether sockets is listening or not,
\item [\texttt{t}] only TCP connections.
\item [\cmdvar] \textbf{ping} tests the reachability of a host 
on an IP network by sending ICMP ECHO\_REQUEST:
\item [\texttt{c}] stops after sending ``count'' packets,
\item [\texttt{n}] numeric output only, avoids symbolic names for host addresses lookup. 
%%% h
\item [\cmdvar] \textbf{route} shows and manipulates the IP routing table.
\item [\cmdvar] \textbf{traceroute} prints a trace of the route that IP packets are travelling to a remote host:
\item [\texttt{I}] uses ICMP ECHO for probes.
\end{compactenum}

%\item [\cmdvar] \textbf{inetd} is a super-server daemon that provides Internet services.
%\item [\cmdvar] \textbf{netcat}: arbitrary TCP and UDP connections and listens.
%\item [\cmdvar] \textbf{rdate} sets the system's date from a remote host.

%\item [\texttt{}] \texttt{----delete} deletes extraneous files from dest dirs.


\begin{compactenum}
\item [\cmdvar] \textbf{rsync} copies files fast (remote or local):
\item [\texttt{a}] in archive mode, equivalent to:
\item [\texttt{g}] preserves group,
\item [\texttt{o}] preserves owner (super-user only)
\item [\texttt{p}] preserves permissions,
\item [\texttt{t}] preserves modification times,
\item [\texttt{l}] copies symlinks as symlinks,
\item [\texttt{b}] make backups, 
\item [\texttt{c}] skip based on checksum, 
\item [\texttt{n}] performs a dry run without changes made, 
\item [\texttt{r}] resursively,  %%%
\item [\texttt{u}] skip newer files on the receiver, 
\item [\texttt{v}] increases verbosity, %%%
\item [\texttt{z}] compresses file data during the transfer, %%%
\item [\cmdvar] \textbf{scp} copies files securely.
\end{compactenum}

\subsection{Searching}
\begin{compactenum}
\item [\cmdvar] \textbf{find} searches for files in a directory hierarchy.
\item [\cmdvar] \textbf{locate} finds files by names.
\item [\cmdvar] \textbf{updatedb} updates the file database used by locate.
\item [\cmdvar] \textbf{whatis} displays one-line manual page description.
\item [\cmdvar] \textbf{whereis} locates the binary, source, 
and manual page files for a command.
\end{compactenum}


\subsection{Hardware}
\begin{compactenum}
\item [\cmdutil] \textbf{dmesg} prints/controls the kernel ring buffer.
\item [\cmdutil] \textbf{lsblk} lists block devices.
\item [\cmdvar] \textbf{lsof} lists info about files opened by processes.
\item [?] \textbf{fuser} identifies processes using files/sockets.
\item [\cmdvar] \textbf{lsusb} listsq USB devices.
\end{compactenum}

\subsection{For programmers}
\begin{compactenum}
	\item [\cmdcore] \textbf{g++} compiles, assembles and links C++ files:
	\item [\texttt{o}] writes the build output to a file named \ldots
\end{compactenum}



\subsection{Miscellaneous}
\begin{compactenum}
\item [\cmdvar] \textbf{bc} is an arbitrary precision calculator language.
\item \texttt{echo 'obase=16;255' | bc} prints \texttt{FF},
\item \texttt{echo 'ibase=2;obase=A;10' | bc} prints \texttt{2},
\item \texttt{scale=10} (after \texttt{bc -l}) sets working precision.
\item [\cmdvar] \textbf{dc} is a reverse-polish desk calculator.
One of the oldest Unix utilities, 
predating even the invention of the C programming language.
\item [\cmdutil] \textbf{cal}, \textbf{ncal} displays a calendar.
\item [\texttt{e}] displays date of Easter,
\item [\texttt{j}] displays Julian days,
\item [\texttt{m}] displays the specified month,
\item [\texttt{w}] prints the numbers of the weeks,
\item [\texttt{y}] displays a calendar for the specified year,
\item [\texttt{3}] displays the previous, current and next month.
\item [\cmdvar] \textbf{date} prints or set the system date and time.
% \textbf{expr}
%\item [\cmdvar] \textbf{lp} prints files.
%\item [\cmdvar] \textbf{od} dumps files in octal.
% hexdump -C, xxd
\item [\cmdcore] \textbf{seq} prints a sequence of numbers:
\item [\texttt{w}] equalizes width by padding with leading zeroes.
\item [\cmdcore] \textbf{sleep} delays for a specified amount of time.
\item [\cmdvar] \textbf{true}, \textbf{false} does nothing, (un)successfully.
\end{compactenum}
\end{multicols}
\end{document}

Todo:
\textbf{gp} invokes the PARI/GP calculator.
\textbf{pdflatex} runs the pdfTeX typesetter.

\textbf{apropos} searches the manual page names and descriptions.
\textbf{man} is an interface to the online reference manuals.

\textbf{ghci} is the Glasgow Haskell Compiler.
\textbf{ipython} is an interactive Python shell, see also
\textbf{python} and \textbf{python3}.
\textbf{gcc} is a C and C++ compiler.

nc - ?
