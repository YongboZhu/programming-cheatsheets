\section{Text processing: grep, sed, awk}
This is an expanded description of three powerful text processing tools:
\texttt{grep}, \texttt{sed} and \texttt{awk}.

\subsection{grep -- pattern search enginge}
The ed command \texttt{g/re/p} was used to \textbf{g}lobally search a \textbf{r}egular \textbf{e}xpression and \textbf{p}rint.
\begin{compactenum}
\item [\cmdvar] \textbf{grep} prints lines matching a pattern:
\item [\texttt{c}] prints a count of matching lines instead,
\item [\texttt{e}] uses a ``regexp'' pattern,
\item [\texttt{f}] obtains patterns from a file,
\item [\texttt{i}] ignores case disctinctions,
\item [\texttt{v}] inverts the sense of matching,
\item [\texttt{w}] selects only lines with whole words matches,
\item [\texttt{n}] prints line numbers as well,
\item [\texttt{A}] prints ``num'' lines of trailing content,
\item [\texttt{B}] prints ``num'' lines of leading content,
\item [\texttt{C}] prints ``num'' lines of both contents,
\item [\texttt{E}] interprets pattern as an extended regexp,
\item [\texttt{P}] interprets pattern as a Perl regexp,
\item [\texttt{R}] reads all files under each directory.
\end{compactenum}

\subsection{sed -- stream editor}
\begin{compactenum}
	\item [\cmdcore] \textbf{sed} filters and transforms text:
	\item [\texttt{-e}] adds a script to the commands to be executed,
	\item [\texttt{-i}] edits files in place,
	\item [\texttt{-n}] suppresses auto- printing of pattern space,
  \item [\texttt{-r}] accepts extended regular expressions.
\end{compactenum}

The simplest usage is \mintinline{bash}{sed 's/foo/bar/g'}
which substitutes (\texttt{s}) strings globally (\texttt{g}).
There are other options, including:
\begin{compactenum}
\item [\texttt{a}] appends line before,
\item [\texttt{d}] deletes line,
\item [\texttt{i}] inserts line before,
\item [\texttt{p}] prints line,
\item [\texttt{w}] writes pattern space to a file.
\end{compactenum}

Default delimiter \texttt{/} can be replaced by any other.
This is useful when regular expression already contains \texttt{/}.
Addresses allow limiting to given line numbers:
\begin{compactenum}
\item \texttt{1-10} first ten lines
\item \texttt{\$} the last line
\item \texttt{10\textasciitilde 2} even lines starting from the 10th.
\end{compactenum}

One can also use regular expressions:
\begin{minted}{bash}
sed -e '/:/s/ /_/g'
\end{minted}
replaces spaces with underscores in lines containing a colon.
Negation may be obtained with \texttt{!s}.

% # d 1d 1,10d

\subsection{awk -- Aho, Weinberger, Kernighan}
\begin{compactenum}
\item [\cmdvar] \textbf{awk} is a language used as a data extraction and reporting tool.
\end{compactenum}

General form of its code:
\begin{minted}{bash}
#! /bin/awk
BEGIN {initialization}
search pattern {actions} # for example:
/word[0-9]/ {gold += $2} # regex
!/word[0-9]/ {counter++} # negation
END {final actions}
\end{minted}

Awk is weakly typed: variables can be treated either as numeric values or strings, which are not represented as one-dimensional arrays of characters!
Important variables include:
\begin{compactenum}
\item \textbf{FS}: field separator (tab and space by default),
\item \textbf{OFS}: output field separator,
\item \textbf{RS}: record separator (new line),
\item \textbf{NR}: number of the current record,
\item \textbf{NF}: number of fields in the current record. 
\end{compactenum}

Numerical functions: \texttt{int}, \texttt{sqrt}, \texttt{exp}, \texttt{log}, \texttt{sin}, \texttt{cos}, \texttt{atan2}, \texttt{rand} (pseudo random from $[0, 1)$), \texttt{srand} (without parameters, uses time of day as a seed).

String/text functions: \texttt{length}, \texttt{split}, \texttt{sprintf}, \texttt{gsub}, \texttt{sub}, \texttt{index}, \texttt{match}, \texttt{tolower}, \texttt{toupper}.