\section{Programming in Bash}
\subsection{Shebang}
The shebang (\texttt{\#!}) at the head of a script indicates an 
interpreter for execution, as in \texttt{\#!/bin/bash}.
Lines starting with a \texttt{\#} (with the exception of shebang) 
are comments and thus won't be executed.

\subsection{Streams}
There are always three default files open:
\begin{enumx}
\item \emph{stdin} (the keyboard, file descriptor 0),
\item \emph{stdout} (the screen, file descriptor 1) and
\item \emph{stderr} (error messages output, file descriptor 2).
\end{enumx}

These \textbf{streams} can be \textbf{redirected}:
\begin{itemx} 
\item \texttt{cmd > file} redirects to a file (overwrites),
\item \texttt{cmd >{}> file} appends instead,
\item \texttt{m>n} (or \texttt{m>\&n}) redirects a file descriptor to a file 
(or another file descriptor), 
\item \texttt{\&>file} redirects both stdout and stderr to a file;
\item \texttt{:> file} truncates file to zero length and
\item \texttt{|} (pipe) serves as a command chaining tool.
\end{itemx}

\subsection{Variables}
\textbf{Variable} names are case sensitive.
They can contain digits and underscores as well,
but a name starting with a digit is not allowed.
Example: 
\begin{minted}{bash}
var="kind"
echo ${var}ness # kindness
\end{minted}

Special variables:
\begin{enumx}
	\item \texttt{\$0}: name of the script itself,
	\item \texttt{\$1}, \ldots: the first, second, etc. argument.
	\item \texttt{\$*} and \texttt{\$@} denote all the positional parameters.
	\item \texttt{\$\#}: the number of positional parameters
	\item \texttt{\$?}: most recently executed command exit status.
	\item \texttt{\$\$}: the process ID of the shell.
	\item \texttt{\$!}: the process ID of the most recently executed command.
\end{enumx}

\subsection{Control flow statements}
\subsubsection{Conditionals}
Here at least one statement must be specified inside every block,
but one can use a single colon (:) as a null statement to avoid
rewriting the code.

\begin{minted}{bash}
if first_condition; then
    commands
elif second_condition; then
    some_commands
else
    other_commands	
fi

case $fruit in
    banana)
        echo "Bananas are awry."
        ;;
    orange|apple)
        echo "Ugh..."
        exit 1
        ;;
    *)
        echo "Unknown fruit!"
esac
\end{minted}

\subsubsection{Loops}
\begin{minted}{bash}
for var in "the first" "the second"; do
    echo "${var}"
done

for (( i = 1; i <= 10; i++ )); do
    echo "i = ${i}."
done # C-style

while read myline; do
    echo "It says ${myline}"
done < some_file
\end{minted}

As Bash Guide for Beginners by M. Garrels says:
\begin{enumx}
\item the \texttt{break} statement is used to 
exit the current loop before its normal ending. 
\item the \texttt{continue} statement resumes iteration 
of an enclosing while, until, select or for loop.
\end{enumx}


% special files: /dev/null, /dev/zero, /proc/...

% special meaning of < > ; | * ? - one has to escape them
% ' '
% " "
