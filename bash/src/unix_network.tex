\begin{compactenum}
\item [\cmdvar] \textbf{curl} transfers a URL.
\item [\cmdvar] \textbf{wget} is a non-interactive network downloader.
\item [\texttt{A}, \texttt{R}] specifies lists 	of file suffixes or 
	patterns (when wildcard characters appear) to accept or reject,
\item [\texttt{b}] goes to background immediately after startup,
\item [\texttt{c}] continues getting a partially-downloaded file,
\item [\texttt{m}] turns on options suitable for mirroring: 
	infinite recursion and time-stamping,
\item [\texttt{np}] does not ever ascend to the
	parent directory when retrieving recursively,
\item [\texttt{U}] identifies as ``agent-string'' to the HTTP server.
\item [\texttt{w}] waits the specified number of seconds 
	between the retrievals (see also \texttt{--random-wait}).
\end{compactenum}

\begin{compactenum}
\item [\cmdvar] \textbf{rlogin} starts a terminal session on a remote host.
\item [\cmdvar] \textbf{ssh} is an OpenSSH SSH client (remote login program).
\item [\texttt{D}] specifies a local ''dynamic'' application-level port forwarding,
\item [\texttt{p}] selects a port to connect to on the remote host,
\item [\texttt{X}] enables X11 forwarding.
\end{compactenum}

\begin{compactenum}
\item [\cmdvar] \textbf{dig} interrogates DNS name servers.                        
\item [\texttt{x}] performs a simplified reverse lookup. 
\item [\cmdvar] \textbf{host} is a DNS lookup utility.  
\item [\cmdvar] \textbf{nslookup} is (probably) deprecated! Use \textbf{dig} and \textbf{host}.
\end{compactenum}

\begin{compactenum}
\item [\cmdvar] \textbf{ifconfig} configures a network interface.   
\item [\cmdvar] \textbf{inetd} is a super-server daemon that provides Internet services.
\item [\cmdvar] \textbf{netcat}: arbitrary TCP and UDP connections and listens.
\item [\cmdvar] \textbf{netstat} prints network connections, routing tables, 
interface statistics, masquerade connections, and multicast memberships.
\item [\cmdvar] \textbf{ping} tests the reachability of a host 
on an IP network by sending ICMP ECHO\_REQUEST:
\item [\texttt{c}] stops after sending ``count'' packets,
\item [\texttt{n}] numeric output only, 
	avoids to lookup symbolic names for host addresses. 
\item [\cmdvar] \textbf{rdate} sets the system's date from a remote host.
\item [\cmdvar] \textbf{rsync} copies files fast (remote or local):
\item [\texttt{a}] in archive mode, equivalent to:
\item [\texttt{g}] preserves group,
\item [\texttt{o}] preserves owner (super-user only)
\item [\texttt{p}] preserves permissions,
\item [\texttt{t}] preserves modification times,
\item [\texttt{l}] copies symlinks as symlinks,
\item [\texttt{b}] make backups, 
\item [\texttt{c}] skip based on checksum, 
\item [\texttt{n}] performs a dry run without changes made, 
\item [\texttt{r}] resursively, 
\item [\texttt{u}] skip newer files on the receiver, 
\item [\texttt{v}] increases verbosity, 
\item [\texttt{z}] compresses file data during the transfer,
\item [\texttt{}] \texttt{----delete} deletes extraneous files from dest dirs.
\item [\cmdvar] \textbf{route} shows and manipulates the IP routing table.
\item [\cmdvar] \textbf{traceroute} is a computer network diagnostic tool for 
displaying the route (path) and measuring transit delays of 
\end{compactenum}
