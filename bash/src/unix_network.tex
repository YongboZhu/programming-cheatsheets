\begin{enumx}
\item [\cmd] \textbf{curl} transfers a URL.
\item [\cmd] \textbf{dig} interrogates DNS name servers.                        
\item [\texttt{x}] performs a simplified reverse lookup. 
\item [\cmd] \textbf{host} is a DNS lookup utility.  
\item [\cmd] \textbf{ifconfig} configures a network interface.   
\item [\cmd] \textbf{inetd} is a super-server daemon that provides Internet services.
\item [\cmd] \textbf{netcat}: arbitrary TCP and UDP connections and listens.
\item [\cmd] \textbf{netstat} prints network connections, routing tables, 
interface statistics, masquerade connections, and multicast memberships.
\item [\cmd] \textbf{nslookup} queries Internet name servers interactively.
\item [\cmd] \textbf{ping} tests the reachability of a host 
on an IP network by sending ICMP ECHO\_REQUEST:
\item [\texttt{c}] stops after sending ``count'' packets,
\item [\texttt{n}] numeric output only, 
	avoids to lookup symbolic names for host addresses. 
\item [\cmd] \textbf{rdate} sets the system's date from a remote host.
\item [\cmd] \textbf{rlogin} starts a terminal session on a remote host.
\item [\cmd] \textbf{route} shows and manipulates the IP routing table.
\item [\cmd] \textbf{ssh} is an OpenSSH SSH client (remote login program).
\item [\texttt{D}] (bind address)
\item [\texttt{p}] (port)
\item [\texttt{X}] (X11 forwarding)
\item [\cmd] \textbf{traceroute} is a computer network diagnostic tool for 
displaying the route (path) and measuring transit delays of 
\item [\cmd] \textbf{wget} is a non-interactive network downloader.
\item [\texttt{A}, \texttt{R}] specifies lists 	of file suffixes or 
	patterns (when wildcard characters appear) to accept or reject,
\item [\texttt{b}] goes to background immediately after startup,
\item [\texttt{c}] continues getting a partially-downloaded file,
\item [\texttt{m}] turns on options suitable for mirroring: 
	infinite recursion and time-stamping,
\item [\texttt{np}] does not ever ascend to the
	parent directory when retrieving recursively,
\item [\texttt{U}] identifies as ``agent-string'' to the HTTP server.
\item [\texttt{w}] waits the specified number of seconds 
	between the retrievals (see also \texttt{--random-wait}).
\end{enumx}
