\begin{compactenum}
\item [\cmdvar] \textbf{bc} is an arbitrary precision calculator language.
\item \texttt{echo 'obase=16;255' | bc} prints \texttt{FF},
\item \texttt{echo 'ibase=2;obase=A;10' | bc} prints \texttt{2},
\item \texttt{scale=10} (after \texttt{bc -l}) sets working precision.
\item [\cmdvar] \textbf{dc} is a reverse-polish desk calculator.
One of the oldest Unix utilities, 
predating even the invention of the C programming language.
\item [\cmdutil] \textbf{cal}, \textbf{ncal} displays a calendar.
\item [\texttt{e}] displays date of Easter,
\item [\texttt{j}] displays Julian days,
\item [\texttt{m}] displays the specified month,
\item [\texttt{w}] prints the numbers of the weeks,
\item [\texttt{y}] displays a calendar for the specified year,
\item [\texttt{3}] displays the previous, current and next month.
\item [\cmdvar] \textbf{date} prints or set the system date and time.
% \textbf{expr}
%\item [\cmdvar] \textbf{lp} prints files.
%\item [\cmdvar] \textbf{od} dumps files in octal.
% hexdump -C, xxd
\item [\cmdcore] \textbf{seq} prints a sequence of numbers:
\item [\texttt{w}] equalizes width by padding with leading zeroes.
\item [\cmdcore] \textbf{sleep} delays for a specified amount of time.
\item [\cmdvar] \textbf{true}, \textbf{false} does nothing, (un)successfully.
\end{compactenum}